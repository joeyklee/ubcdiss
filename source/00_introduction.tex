\chapter{Introduction}
\label{ch:Introduction}

Lorem ipsum dolor sit amet, consectetur adipiscing elit, sed do eiusmod tempor incididunt ut labore et dolore magna aliqua. Ut enim ad minim veniam, quis nostrud exercitation ullamco laboris nisi ut aliquip ex ea commodo consequat. Duis aute irure dolor in reprehenderit in voluptate velit esse cillum dolore eu fugiat nulla pariatur. Excepteur sint occaecat cupidatat non proident, sunt in culpa qui officia deserunt mollit anim id est laborum.


\begin{figure}[!htbp]
	\centering
	\includegraphics[width=0.8\textwidth]{figures/Nyan_Cat_Wide.jpg} 
	\caption[This is a nyan cat caption]{
		Insert figure caption here.
	}
	\label{fig:nyan-cat-figure}
\end{figure}\FloatBarrier

Lorem ipsum dolor sit amet, consectetur adipiscing elit, sed do eiusmod tempor incididunt ut labore et dolore magna aliqua. Ut enim ad minim veniam, quis nostrud exercitation ullamco laboris nisi ut aliquip ex ea commodo consequat. Duis aute irure dolor in reprehenderit in voluptate velit esse cillum dolore eu fugiat nulla pariatur. Excepteur sint occaecat cupidatat non proident, sunt in culpa qui officia deserunt mollit anim id est laborum.

% ------------------------------------------------------- %
\section{State of the Art}
\label{StateOfTheArt}

Lorem ipsum dolor sit amet, consectetur adipiscing elit, sed do eiusmod tempor incididunt ut labore et dolore magna aliqua. Ut enim ad minim veniam, quis nostrud exercitation ullamco laboris nisi ut aliquip ex ea commodo consequat. Duis aute irure dolor in reprehenderit in voluptate velit esse cillum dolore eu fugiat nulla pariatur. Excepteur sint occaecat cupidatat non proident, sunt in culpa qui officia deserunt mollit anim id est laborum.


% ------------------------------------------------------- %
\subsection{This is a subsection }
\label{sec:subsection1}

Lorem ipsum dolor sit amet, consectetur adipiscing elit, sed do eiusmod tempor incididunt ut labore et dolore magna aliqua. Ut enim ad minim veniam, quis nostrud exercitation ullamco laboris nisi ut aliquip ex ea commodo consequat. Duis aute irure dolor in reprehenderit in voluptate velit esse cillum dolore eu fugiat nulla pariatur. Excepteur sint occaecat cupidatat non proident, sunt in culpa qui officia deserunt mollit anim id est laborum.

% \citealp{Kellett:2013fk, Christen:2011gu}. 
% \citep{Kellett:2013fk}

\begin{table}[htbp]
\centering
    \caption[Table of measurement methods at various urban scales]{Relevant scales and monitoring methods in urban settings. Adapted from \citep{}}.
    \label{table:scales}
\begin{tabular}{L{2cm}L{1cm}L{2cm}L{3cm}L{3cm}} 
 \toprule 
Scale (Systems studies)  & Spatial Dimension & Atmospheric Layer Studied  & Temporal Resolution & Common Measurement Approaches\\
\midrule 
Micro-scale (buildings, roads, industry, greenspace) & 1-100 m  & urban canopy layer, roughness sublayer & One-time measurements at specific locations or along transects, in some cases long-term observations (5 min to years). & Traverse and vertical profile measurements in canyons, ecophysiological measurements using closed-chambers (vegetation, soils). \\
Local-scale (neighborhoods, land-use zones)          & 100 - 10 km       & internal sublayer                      & Continuous measurements that resovle diurnal and seasonal dynamics (30 min to years).                                         & Direct eddy-covariance flux measurements on towers above the city.                                                              \\
Meso-scale (cities, urban regions)                   & 10 - 100 km       & urban boundary layer                   & Short-term campaigns or continuous measurements at selected sites that resolve day-today variations and seasonal differences. & Boundary-layer budgets, upwind-downwind mixing ratio differences, regional inverse modelling, isotopic ratios.  \\            
\bottomrule 
\end{tabular} 
\end{table}
\FloatBarrier

% ------------------------------------------------------- %
\subsection{This is a subsection }
\label{sec:subsection2}


\subsubsection{This is a sub subsection}
\label{sec:subsubsection1}


\subsubsection{Objectives}
In order to address the research question, five major objectives were outlined and developed:
\begin{enumerate}
	\item Sensor Development: Develop and test a compact, mobile, and multi-modal $\rm{CO}_2$ sensor for bikes and cars.
	\item Measurement Campaign: Deploy the sensors in a targeted measurement campaign.
	\item Physical Concept: Develop a methodology to calculate emissions from measurements of $\rm{CO}_2$ mixing ratios using knowledge about atmospheric conditions. 
	\item Analysis and Evaluation: Compare the mixing ratio measurements and measured emissions to traffic and building emissions inventories. 
	% \item Visualization: Develop an interactive visualization that allows exploration of the data and concepts of the methodology and results. 
\end{enumerate}

